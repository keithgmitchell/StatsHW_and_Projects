\PassOptionsToPackage{unicode=true}{hyperref} % options for packages loaded elsewhere
\PassOptionsToPackage{hyphens}{url}
%
\documentclass[]{article}
\usepackage{lmodern}
\usepackage{amssymb,amsmath}
\usepackage{ifxetex,ifluatex}
\usepackage{fixltx2e} % provides \textsubscript
\ifnum 0\ifxetex 1\fi\ifluatex 1\fi=0 % if pdftex
  \usepackage[T1]{fontenc}
  \usepackage[utf8]{inputenc}
  \usepackage{textcomp} % provides euro and other symbols
\else % if luatex or xelatex
  \usepackage{unicode-math}
  \defaultfontfeatures{Ligatures=TeX,Scale=MatchLowercase}
\fi
% use upquote if available, for straight quotes in verbatim environments
\IfFileExists{upquote.sty}{\usepackage{upquote}}{}
% use microtype if available
\IfFileExists{microtype.sty}{%
\usepackage[]{microtype}
\UseMicrotypeSet[protrusion]{basicmath} % disable protrusion for tt fonts
}{}
\IfFileExists{parskip.sty}{%
\usepackage{parskip}
}{% else
\setlength{\parindent}{0pt}
\setlength{\parskip}{6pt plus 2pt minus 1pt}
}
\usepackage{hyperref}
\hypersetup{
            pdftitle={Final Project},
            pdfauthor={Keith Mitchell},
            pdfborder={0 0 0},
            breaklinks=true}
\urlstyle{same}  % don't use monospace font for urls
\usepackage[margin=1in]{geometry}
\usepackage{color}
\usepackage{fancyvrb}
\newcommand{\VerbBar}{|}
\newcommand{\VERB}{\Verb[commandchars=\\\{\}]}
\DefineVerbatimEnvironment{Highlighting}{Verbatim}{commandchars=\\\{\}}
% Add ',fontsize=\small' for more characters per line
\usepackage{framed}
\definecolor{shadecolor}{RGB}{248,248,248}
\newenvironment{Shaded}{\begin{snugshade}}{\end{snugshade}}
\newcommand{\AlertTok}[1]{\textcolor[rgb]{0.94,0.16,0.16}{#1}}
\newcommand{\AnnotationTok}[1]{\textcolor[rgb]{0.56,0.35,0.01}{\textbf{\textit{#1}}}}
\newcommand{\AttributeTok}[1]{\textcolor[rgb]{0.77,0.63,0.00}{#1}}
\newcommand{\BaseNTok}[1]{\textcolor[rgb]{0.00,0.00,0.81}{#1}}
\newcommand{\BuiltInTok}[1]{#1}
\newcommand{\CharTok}[1]{\textcolor[rgb]{0.31,0.60,0.02}{#1}}
\newcommand{\CommentTok}[1]{\textcolor[rgb]{0.56,0.35,0.01}{\textit{#1}}}
\newcommand{\CommentVarTok}[1]{\textcolor[rgb]{0.56,0.35,0.01}{\textbf{\textit{#1}}}}
\newcommand{\ConstantTok}[1]{\textcolor[rgb]{0.00,0.00,0.00}{#1}}
\newcommand{\ControlFlowTok}[1]{\textcolor[rgb]{0.13,0.29,0.53}{\textbf{#1}}}
\newcommand{\DataTypeTok}[1]{\textcolor[rgb]{0.13,0.29,0.53}{#1}}
\newcommand{\DecValTok}[1]{\textcolor[rgb]{0.00,0.00,0.81}{#1}}
\newcommand{\DocumentationTok}[1]{\textcolor[rgb]{0.56,0.35,0.01}{\textbf{\textit{#1}}}}
\newcommand{\ErrorTok}[1]{\textcolor[rgb]{0.64,0.00,0.00}{\textbf{#1}}}
\newcommand{\ExtensionTok}[1]{#1}
\newcommand{\FloatTok}[1]{\textcolor[rgb]{0.00,0.00,0.81}{#1}}
\newcommand{\FunctionTok}[1]{\textcolor[rgb]{0.00,0.00,0.00}{#1}}
\newcommand{\ImportTok}[1]{#1}
\newcommand{\InformationTok}[1]{\textcolor[rgb]{0.56,0.35,0.01}{\textbf{\textit{#1}}}}
\newcommand{\KeywordTok}[1]{\textcolor[rgb]{0.13,0.29,0.53}{\textbf{#1}}}
\newcommand{\NormalTok}[1]{#1}
\newcommand{\OperatorTok}[1]{\textcolor[rgb]{0.81,0.36,0.00}{\textbf{#1}}}
\newcommand{\OtherTok}[1]{\textcolor[rgb]{0.56,0.35,0.01}{#1}}
\newcommand{\PreprocessorTok}[1]{\textcolor[rgb]{0.56,0.35,0.01}{\textit{#1}}}
\newcommand{\RegionMarkerTok}[1]{#1}
\newcommand{\SpecialCharTok}[1]{\textcolor[rgb]{0.00,0.00,0.00}{#1}}
\newcommand{\SpecialStringTok}[1]{\textcolor[rgb]{0.31,0.60,0.02}{#1}}
\newcommand{\StringTok}[1]{\textcolor[rgb]{0.31,0.60,0.02}{#1}}
\newcommand{\VariableTok}[1]{\textcolor[rgb]{0.00,0.00,0.00}{#1}}
\newcommand{\VerbatimStringTok}[1]{\textcolor[rgb]{0.31,0.60,0.02}{#1}}
\newcommand{\WarningTok}[1]{\textcolor[rgb]{0.56,0.35,0.01}{\textbf{\textit{#1}}}}
\usepackage{graphicx,grffile}
\makeatletter
\def\maxwidth{\ifdim\Gin@nat@width>\linewidth\linewidth\else\Gin@nat@width\fi}
\def\maxheight{\ifdim\Gin@nat@height>\textheight\textheight\else\Gin@nat@height\fi}
\makeatother
% Scale images if necessary, so that they will not overflow the page
% margins by default, and it is still possible to overwrite the defaults
% using explicit options in \includegraphics[width, height, ...]{}
\setkeys{Gin}{width=\maxwidth,height=\maxheight,keepaspectratio}
\setlength{\emergencystretch}{3em}  % prevent overfull lines
\providecommand{\tightlist}{%
  \setlength{\itemsep}{0pt}\setlength{\parskip}{0pt}}
\setcounter{secnumdepth}{0}
% Redefines (sub)paragraphs to behave more like sections
\ifx\paragraph\undefined\else
\let\oldparagraph\paragraph
\renewcommand{\paragraph}[1]{\oldparagraph{#1}\mbox{}}
\fi
\ifx\subparagraph\undefined\else
\let\oldsubparagraph\subparagraph
\renewcommand{\subparagraph}[1]{\oldsubparagraph{#1}\mbox{}}
\fi

% set default figure placement to htbp
\makeatletter
\def\fps@figure{htbp}
\makeatother


\title{Final Project}
\author{Keith Mitchell}
\date{3/7/2020}

\begin{document}
\maketitle

\url{https://www.chegg.com/homework-help/Applied-Multivariate-Statistical-Analysis-6th-edition-chapter-11-problem-32E-solution-9780131877153}
\url{https://www.chegg.com/homework-help/Applied-Multivariate-Statistical-Analysis-6th-edition-chapter-8-problem-11E-solution-9780131877153}

\begin{center}
STA135 Final Project (UC Davis)
\end{center}

\begin{center}
Instructor: Professor Li
\end{center}

\begin{center}
TA: Cong Xu
\end{center}

\begin{center}\rule{0.5\linewidth}{0.5pt}\end{center}

\hypertarget{dataset-1-conduct-multiple-linear-regression}{%
\subsection{Dataset 1: Conduct multiple linear
regression;}\label{dataset-1-conduct-multiple-linear-regression}}

\hypertarget{dataset-2-conduct-two-sample-test-and-lda}{%
\subsection{Dataset 2: Conduct two-sample test and
LDA;}\label{dataset-2-conduct-two-sample-test-and-lda}}

\hypertarget{dataset-3-conduct-pca.}{%
\subsection{Dataset 3: Conduct PCA.}\label{dataset-3-conduct-pca.}}

\hypertarget{for-each-data-analysis-you-should-write-in-full-sentences-and-have-thefollowing-sections-for-the-body-of-your-report.}{%
\subsubsection{For each data analysis, you should write in full
sentences, and have thefollowing sections for the body of your
report.}\label{for-each-data-analysis-you-should-write-in-full-sentences-and-have-thefollowing-sections-for-the-body-of-your-report.}}

\begin{enumerate}
\def\labelenumi{\arabic{enumi})}
\tightlist
\item
  Introduction: Briefly summarize the goal of the analysis in your own
  words;
\item
  Summary: Summarize your data by plots or sample estimates;
\item
  Analysis: Implement the analysis based on what you have done in
  homework;
\item
  Conclusion: Describe and interpret your findings.
\end{enumerate}

\hypertarget{details}{%
\subsubsection{Details:}\label{details}}

\begin{enumerate}
\def\labelenumi{\arabic{enumi})}
\tightlist
\item
  A title page including your name, the name of the class, and the name
  of your instructor.
\item
  Do not include R code in the body of your report. R code used to
  produce the results should all go to the appendix.(echo=FALSE)
\item
  Typed.
\item
  Double-sided pages.
\end{enumerate}

\begin{center}\rule{0.5\linewidth}{0.5pt}\end{center}

\newpage

\newpage

\begin{center}\rule{0.5\linewidth}{0.5pt}\end{center}

\hypertarget{dataset-2-two-sample-test-and-lda}{%
\section{DATASET 2: Two Sample Test and
LDA}\label{dataset-2-two-sample-test-and-lda}}

\hypertarget{introduction}{%
\subsection{INTRODUCTION:}\label{introduction}}

\hypertarget{data-has-been-gather-on-hemophilia-a-carriers.}{%
\subsubsection{Data has been gather on hemophilia A
carriers.}\label{data-has-been-gather-on-hemophilia-a-carriers.}}

\begin{enumerate}
\def\labelenumi{\arabic{enumi})}
\tightlist
\item
  Predicting disease based on other characteristics is a common
  technique that doctors and healthcare workers use to produce a
  prediction for a given person.
\item
  This is a common method that if not used solely for diagnosing a
  disease is used in conjunction with other tests to increase the
  statistical power of the diagnoses. \#\#\# The first goal of this
  analysis will be to perform two sample test on the data in order to
  see if there is a significant difference between individuals that are
  carriers and noncarriers with respect to AHF activity (\(log_{10}\)
  scale) and AHF antigen (\(log_{10}\) scale). Group 1 is considered to
  be the non carrier group while group 2 is considered to be the
  obligatory carrier group. \#\#\# The second goal of this analysis will
  be to perform linear discriminant analysis to try and predict the best
  possible distinguishable boundary between carriers and non carriers
  with respect to AHF activity (\(log_{10}\) scale) and AHF antigen
  (\(log_{10}\) scale).
\end{enumerate}

\hypertarget{summarize-data}{%
\subsection{SUMMARIZE DATA:}\label{summarize-data}}

\hypertarget{lets-look-at-the-data-as-a-whole-with-the-two-groups-together-carriers-and-non-carriers}{%
\subsubsection{Lets look at the data as a whole (with the two groups
together (carriers and non
carriers))}\label{lets-look-at-the-data-as-a-whole-with-the-two-groups-together-carriers-and-non-carriers}}

\begin{verbatim}
## [1] 75  2
\end{verbatim}

\begin{verbatim}
##          Group     AHF_activity      AHF_antigen      
##  Non-carrier:30   Min.   :-0.6911   Min.   :-0.47730  
##  Carrier    :45   1st Qu.:-0.3609   1st Qu.:-0.12110  
##                   Median :-0.2015   Median :-0.04070  
##                   Mean   :-0.2387   Mean   :-0.03474  
##                   3rd Qu.:-0.1176   3rd Qu.: 0.07520  
##                   Max.   : 0.1507   Max.   : 0.28760
\end{verbatim}

\includegraphics{STA135-FinalProject_files/figure-latex/unnamed-chunk-11-1.pdf}

\includegraphics{STA135-FinalProject_files/figure-latex/unnamed-chunk-12-1.pdf}
\includegraphics{STA135-FinalProject_files/figure-latex/unnamed-chunk-12-2.pdf}

\begin{verbatim}
## [1] "Summary and dimensions of the first group of individuals."
\end{verbatim}

\begin{verbatim}
##          Group     AHF_activity       AHF_antigen      
##  Non-carrier:30   Min.   :-0.52680   Min.   :-0.47730  
##  Carrier    : 0   1st Qu.:-0.19470   1st Qu.:-0.14968  
##                   Median :-0.13890   Median :-0.06775  
##                   Mean   :-0.13487   Mean   :-0.07786  
##                   3rd Qu.:-0.07805   3rd Qu.: 0.00955  
##                   Max.   : 0.15070   Max.   : 0.21400
\end{verbatim}

\begin{verbatim}
## [1] 30  3
\end{verbatim}

\begin{verbatim}
## [1] "Summary and dimensions of the second group of individuals."
\end{verbatim}

\begin{verbatim}
##          Group     AHF_activity      AHF_antigen       
##  Non-carrier: 0   Min.   :-0.6911   Min.   :-0.339000  
##  Carrier    :45   1st Qu.:-0.4055   1st Qu.:-0.107900  
##                   Median :-0.3352   Median : 0.004600  
##                   Mean   :-0.3079   Mean   :-0.005991  
##                   3rd Qu.:-0.1878   3rd Qu.: 0.115100  
##                   Max.   :-0.0149   Max.   : 0.287600
\end{verbatim}

\begin{verbatim}
## [1] 45  3
\end{verbatim}

\hypertarget{data-analysis}{%
\subsection{DATA ANALYSIS:}\label{data-analysis}}

\hypertarget{computing-the-lda}{%
\subsubsection{Computing the LDA}\label{computing-the-lda}}

\begin{Shaded}
\begin{Highlighting}[]
\NormalTok{group_}\DecValTok{1}\NormalTok{ <-}\StringTok{ }\NormalTok{group_}\DecValTok{1}\NormalTok{[}\DecValTok{2}\OperatorTok{:}\DecValTok{3}\NormalTok{]}
\NormalTok{group_}\DecValTok{2}\NormalTok{ <-}\StringTok{ }\NormalTok{group_}\DecValTok{2}\NormalTok{[}\DecValTok{2}\OperatorTok{:}\DecValTok{3}\NormalTok{]}

\NormalTok{x1 <-}\StringTok{ }\NormalTok{group_}\DecValTok{1}
\NormalTok{x2 <-}\StringTok{ }\NormalTok{group_}\DecValTok{2}


\CommentTok{# compute sample mean vectors:}

\NormalTok{x1.mean <-}\StringTok{ }\KeywordTok{colMeans}\NormalTok{(x1)}
\NormalTok{x2.mean <-}\StringTok{ }\KeywordTok{colMeans}\NormalTok{(x2)}
\NormalTok{x1.mean}
\end{Highlighting}
\end{Shaded}

\begin{verbatim}
## AHF_activity  AHF_antigen 
##  -0.13487000  -0.07785667
\end{verbatim}

\begin{Shaded}
\begin{Highlighting}[]
\NormalTok{x2.mean}
\end{Highlighting}
\end{Shaded}

\begin{verbatim}
## AHF_activity  AHF_antigen 
## -0.307946667 -0.005991111
\end{verbatim}

\begin{Shaded}
\begin{Highlighting}[]
\CommentTok{# compute pooled estimate for the covariance matrix:}
\CommentTok{#TODO check this formula again}
\NormalTok{S_carrier <-}\StringTok{ }\KeywordTok{var}\NormalTok{(x2)}
\NormalTok{S_noncarrier <-}\StringTok{ }\KeywordTok{var}\NormalTok{(x1)}
\NormalTok{S.u <-}\StringTok{ }\NormalTok{(}\DecValTok{44}\OperatorTok{*}\NormalTok{(}\KeywordTok{var}\NormalTok{(x1))}\OperatorTok{+}\DecValTok{29}\OperatorTok{*}\NormalTok{(}\KeywordTok{var}\NormalTok{(x2)))}\OperatorTok{/}\DecValTok{73}
\NormalTok{w <-}\StringTok{ }\KeywordTok{solve}\NormalTok{(S.u)}\OperatorTok\NormalTok{(x1.mean}\OperatorTok{-}\NormalTok{x2.mean)}
\NormalTok{w0 <-}\StringTok{ }\OperatorTok{-}\NormalTok{(x1.mean}\OperatorTok{+}\NormalTok{x2.mean)}\OperatorTok\NormalTok{w}\OperatorTok{/}\DecValTok{2}
\KeywordTok{ggplot}\NormalTok{(data, }\KeywordTok{aes}\NormalTok{(}\DataTypeTok{x=}\NormalTok{AHF_activity, }\DataTypeTok{y=}\NormalTok{AHF_antigen, }\DataTypeTok{color=}\NormalTok{Group)) }\OperatorTok{+}\StringTok{ }\KeywordTok{geom_point}\NormalTok{() }\OperatorTok{+}\StringTok{ }\KeywordTok{geom_line}\NormalTok{(}\KeywordTok{aes}\NormalTok{(X[,}\DecValTok{1}\NormalTok{], }\OperatorTok{-}\NormalTok{(w[}\DecValTok{1}\NormalTok{]}\OperatorTok{*}\NormalTok{X[,}\DecValTok{1}\NormalTok{]}\OperatorTok{+}\NormalTok{w0)}\OperatorTok{/}\NormalTok{w[}\DecValTok{2}\NormalTok{]))}
\end{Highlighting}
\end{Shaded}

\begin{verbatim}
## Warning in w[1] * X[, 1] + w0: Recycling array of length 1 in vector-array arithmetic is deprecated.
##   Use c() or as.vector() instead.
\end{verbatim}

\includegraphics{STA135-FinalProject_files/figure-latex/unnamed-chunk-14-1.pdf}

\hypertarget{partial-code-credit-to-prof-li-uc-davis-weiping-zhangustc}{%
\section{Partial code credit to Prof Li (UC Davis), Weiping
Zhang(USTC)}\label{partial-code-credit-to-prof-li-uc-davis-weiping-zhangustc}}

\hypertarget{two-sample-test-with}{%
\subsection{Two sample test with}\label{two-sample-test-with}}

\(H_0 : \overrightarrow{\mu}_{noncarrier}\ = \overrightarrow{\mu}_{carrier}\)
at level of \(\alpha\) = 0.05 with the Hotelling's \(T^2\) test.

\begin{Shaded}
\begin{Highlighting}[]
\CommentTok{# now we perform the two-sample Hotelling T^2-test}
\NormalTok{n<-}\KeywordTok{c}\NormalTok{(}\DecValTok{45}\NormalTok{,}\DecValTok{30}\NormalTok{)}
\NormalTok{p<-}\DecValTok{2}
\NormalTok{xmean1<-x1.mean}
\NormalTok{xmean2<-x2.mean}
\NormalTok{d<-xmean1}\OperatorTok{-}\NormalTok{xmean2}
\NormalTok{d}
\end{Highlighting}
\end{Shaded}

\begin{verbatim}
## AHF_activity  AHF_antigen 
##   0.17307667  -0.07186556
\end{verbatim}

\begin{Shaded}
\begin{Highlighting}[]
\NormalTok{Sp<-S.u}
\NormalTok{t2 <-}\StringTok{ }\KeywordTok{t}\NormalTok{(d)}\OperatorTok\KeywordTok{solve}\NormalTok{(}\KeywordTok{sum}\NormalTok{(}\DecValTok{1}\OperatorTok{/}\NormalTok{n)}\OperatorTok{*}\NormalTok{Sp)}\OperatorTok\NormalTok{d}
\NormalTok{t2}
\end{Highlighting}
\end{Shaded}

\begin{verbatim}
##          [,1]
## [1,] 95.16565
\end{verbatim}

\begin{Shaded}
\begin{Highlighting}[]
\NormalTok{alpha<-}\FloatTok{0.05}
\NormalTok{cval <-}\StringTok{ }\NormalTok{(}\KeywordTok{sum}\NormalTok{(n)}\OperatorTok{-}\DecValTok{2}\NormalTok{)}\OperatorTok{*}\NormalTok{p}\OperatorTok{/}\NormalTok{(}\KeywordTok{sum}\NormalTok{(n)}\OperatorTok{-}\NormalTok{p}\DecValTok{-1}\NormalTok{)}\OperatorTok{*}\KeywordTok{qf}\NormalTok{(}\DecValTok{1}\OperatorTok{-}\NormalTok{alpha,p,}\KeywordTok{sum}\NormalTok{(n)}\OperatorTok{-}\NormalTok{p}\DecValTok{-1}\NormalTok{)}
\NormalTok{cval}
\end{Highlighting}
\end{Shaded}

\begin{verbatim}
## [1] 6.33459
\end{verbatim}

\hypertarget{since-t2-95.16-6.33-the-null-hypothesis-is-rejected-at-5-level-of-significance.}{%
\subsection{\texorpdfstring{Since \(T^2\) = 95.16 \textgreater{} 6.33
the null hypothesis is rejected at 5\% level of
significance.}{Since T\^{}2 = 95.16 \textgreater{} 6.33 the null hypothesis is rejected at 5\% level of significance.}}\label{since-t2-95.16-6.33-the-null-hypothesis-is-rejected-at-5-level-of-significance.}}

\hypertarget{confidence-region-for-non-carriers}{%
\subsection{Confidence Region for Non
Carriers}\label{confidence-region-for-non-carriers}}

\begin{Shaded}
\begin{Highlighting}[]
\NormalTok{es<-}\KeywordTok{eigen}\NormalTok{(}\KeywordTok{sum}\NormalTok{(}\DecValTok{1}\OperatorTok{/}\NormalTok{n)}\OperatorTok{*}\NormalTok{Sp)}
\NormalTok{e1<-es}\OperatorTok{$}\NormalTok{vec }\OperatorTok\StringTok{ }\KeywordTok{diag}\NormalTok{(}\KeywordTok{sqrt}\NormalTok{(es}\OperatorTok{$}\NormalTok{val))}
\NormalTok{r1<-}\KeywordTok{sqrt}\NormalTok{(cval)}
\NormalTok{theta<-}\KeywordTok{seq}\NormalTok{(}\DecValTok{0}\NormalTok{,}\DecValTok{2}\OperatorTok{*}\NormalTok{pi,}\DataTypeTok{len=}\DecValTok{250}\NormalTok{)}
\NormalTok{v1<-}\KeywordTok{cbind}\NormalTok{(r1}\OperatorTok{*}\KeywordTok{cos}\NormalTok{(theta), r1}\OperatorTok{*}\KeywordTok{sin}\NormalTok{(theta))}
\NormalTok{pts<-}\KeywordTok{t}\NormalTok{(xmean1}\OperatorTok{-}\NormalTok{(e1}\OperatorTok\KeywordTok{t}\NormalTok{(v1)))}
\KeywordTok{plot}\NormalTok{(pts,}\DataTypeTok{type=}\StringTok{"l"}\NormalTok{,}\DataTypeTok{main=}\StringTok{"Confidence Region for Bivariate Normal"}\NormalTok{,}\DataTypeTok{xlab=}\KeywordTok{expression}\NormalTok{(}\KeywordTok{paste}\NormalTok{(mu, }\StringTok{"(1)"}\NormalTok{)), }\DataTypeTok{ylab=}\KeywordTok{expression}\NormalTok{(}\KeywordTok{paste}\NormalTok{(mu, }\StringTok{"(2)"}\NormalTok{)),}\DataTypeTok{asp=}\DecValTok{1}\NormalTok{)}
\KeywordTok{segments}\NormalTok{(}\DecValTok{0}\NormalTok{,d[}\DecValTok{2}\NormalTok{],d[}\DecValTok{1}\NormalTok{],d[}\DecValTok{2}\NormalTok{],}\DataTypeTok{lty=}\DecValTok{2}\NormalTok{) }\CommentTok{# highlight the center}
\KeywordTok{segments}\NormalTok{(d[}\DecValTok{1}\NormalTok{],}\DecValTok{0}\NormalTok{,d[}\DecValTok{1}\NormalTok{],d[}\DecValTok{2}\NormalTok{],}\DataTypeTok{lty=}\DecValTok{2}\NormalTok{)}
\CommentTok{#TODO check why these bars in the middle are weird and what are the values here corresponding to because we have the two groups to our variables}
\KeywordTok{points}\NormalTok{(X)}

\NormalTok{th2<-}\KeywordTok{c}\NormalTok{(}\DecValTok{0}\NormalTok{,pi}\OperatorTok{/}\DecValTok{2}\NormalTok{,pi,}\DecValTok{3}\OperatorTok{*}\NormalTok{pi}\OperatorTok{/}\DecValTok{2}\NormalTok{,}\DecValTok{2}\OperatorTok{*}\NormalTok{pi)   }\CommentTok{#adding the axis}
\NormalTok{v2<-}\KeywordTok{cbind}\NormalTok{(r1}\OperatorTok{*}\KeywordTok{cos}\NormalTok{(th2), r1}\OperatorTok{*}\KeywordTok{sin}\NormalTok{(th2))}
\NormalTok{pts2<-}\KeywordTok{t}\NormalTok{(d}\OperatorTok{-}\NormalTok{(e1}\OperatorTok\KeywordTok{t}\NormalTok{(v2)))}
\KeywordTok{segments}\NormalTok{(pts2[}\DecValTok{3}\NormalTok{,}\DecValTok{1}\NormalTok{],pts2[}\DecValTok{3}\NormalTok{,}\DecValTok{2}\NormalTok{],pts2[}\DecValTok{1}\NormalTok{,}\DecValTok{1}\NormalTok{],pts2[}\DecValTok{1}\NormalTok{,}\DecValTok{2}\NormalTok{],}\DataTypeTok{lty=}\DecValTok{3}\NormalTok{)  }
\KeywordTok{segments}\NormalTok{(pts2[}\DecValTok{2}\NormalTok{,}\DecValTok{1}\NormalTok{],pts2[}\DecValTok{2}\NormalTok{,}\DecValTok{2}\NormalTok{],pts2[}\DecValTok{4}\NormalTok{,}\DecValTok{1}\NormalTok{],pts2[}\DecValTok{4}\NormalTok{,}\DecValTok{2}\NormalTok{],}\DataTypeTok{lty=}\DecValTok{3}\NormalTok{)}
\end{Highlighting}
\end{Shaded}

\includegraphics{STA135-FinalProject_files/figure-latex/unnamed-chunk-16-1.pdf}

\begin{Shaded}
\begin{Highlighting}[]
\CommentTok{# since we reject the null, we use the simultaneous confidence intervals}
\CommentTok{# to check the significant components}
\end{Highlighting}
\end{Shaded}

\begin{Shaded}
\begin{Highlighting}[]
\NormalTok{es<-}\KeywordTok{eigen}\NormalTok{(}\KeywordTok{sum}\NormalTok{(}\DecValTok{1}\OperatorTok{/}\NormalTok{n)}\OperatorTok{*}\NormalTok{Sp)}
\NormalTok{e1<-es}\OperatorTok{$}\NormalTok{vec }\OperatorTok\StringTok{ }\KeywordTok{diag}\NormalTok{(}\KeywordTok{sqrt}\NormalTok{(es}\OperatorTok{$}\NormalTok{val))}
\NormalTok{r1<-}\KeywordTok{sqrt}\NormalTok{(cval)}
\NormalTok{theta<-}\KeywordTok{seq}\NormalTok{(}\DecValTok{0}\NormalTok{,}\DecValTok{2}\OperatorTok{*}\NormalTok{pi,}\DataTypeTok{len=}\DecValTok{250}\NormalTok{)}
\NormalTok{v1<-}\KeywordTok{cbind}\NormalTok{(r1}\OperatorTok{*}\KeywordTok{cos}\NormalTok{(theta), r1}\OperatorTok{*}\KeywordTok{sin}\NormalTok{(theta))}
\NormalTok{pts<-}\KeywordTok{t}\NormalTok{(d}\OperatorTok{-}\NormalTok{(e1}\OperatorTok\KeywordTok{t}\NormalTok{(v1)))}
\KeywordTok{plot}\NormalTok{(pts,}\DataTypeTok{type=}\StringTok{"l"}\NormalTok{,}\DataTypeTok{main=}\StringTok{"Confidence Region for Bivariate Normal"}\NormalTok{,}\DataTypeTok{xlab=}\KeywordTok{expression}\NormalTok{(}\KeywordTok{paste}\NormalTok{(mu, }\StringTok{"(1)"}\NormalTok{)), }\DataTypeTok{ylab=}\KeywordTok{expression}\NormalTok{(}\KeywordTok{paste}\NormalTok{(mu, }\StringTok{"(2)"}\NormalTok{)),}\DataTypeTok{asp=}\DecValTok{1}\NormalTok{)}
\KeywordTok{segments}\NormalTok{(}\DecValTok{0}\NormalTok{,d[}\DecValTok{2}\NormalTok{],d[}\DecValTok{1}\NormalTok{],d[}\DecValTok{2}\NormalTok{],}\DataTypeTok{lty=}\DecValTok{2}\NormalTok{) }\CommentTok{# highlight the center}
\KeywordTok{segments}\NormalTok{(d[}\DecValTok{1}\NormalTok{],}\DecValTok{0}\NormalTok{,d[}\DecValTok{1}\NormalTok{],d[}\DecValTok{2}\NormalTok{],}\DataTypeTok{lty=}\DecValTok{2}\NormalTok{)}
\CommentTok{#TODO check why these bars in the middle are weird and what are the values here corresponding to because we have the two groups to our variables}
\KeywordTok{points}\NormalTok{(X)}

\NormalTok{th2<-}\KeywordTok{c}\NormalTok{(}\DecValTok{0}\NormalTok{,pi}\OperatorTok{/}\DecValTok{2}\NormalTok{,pi,}\DecValTok{3}\OperatorTok{*}\NormalTok{pi}\OperatorTok{/}\DecValTok{2}\NormalTok{,}\DecValTok{2}\OperatorTok{*}\NormalTok{pi)   }\CommentTok{#adding the axis}
\NormalTok{v2<-}\KeywordTok{cbind}\NormalTok{(r1}\OperatorTok{*}\KeywordTok{cos}\NormalTok{(th2), r1}\OperatorTok{*}\KeywordTok{sin}\NormalTok{(th2))}
\NormalTok{pts2<-}\KeywordTok{t}\NormalTok{(d}\OperatorTok{-}\NormalTok{(e1}\OperatorTok\KeywordTok{t}\NormalTok{(v2)))}
\KeywordTok{segments}\NormalTok{(pts2[}\DecValTok{3}\NormalTok{,}\DecValTok{1}\NormalTok{],pts2[}\DecValTok{3}\NormalTok{,}\DecValTok{2}\NormalTok{],pts2[}\DecValTok{1}\NormalTok{,}\DecValTok{1}\NormalTok{],pts2[}\DecValTok{1}\NormalTok{,}\DecValTok{2}\NormalTok{],}\DataTypeTok{lty=}\DecValTok{3}\NormalTok{)  }
\KeywordTok{segments}\NormalTok{(pts2[}\DecValTok{2}\NormalTok{,}\DecValTok{1}\NormalTok{],pts2[}\DecValTok{2}\NormalTok{,}\DecValTok{2}\NormalTok{],pts2[}\DecValTok{4}\NormalTok{,}\DecValTok{1}\NormalTok{],pts2[}\DecValTok{4}\NormalTok{,}\DecValTok{2}\NormalTok{],}\DataTypeTok{lty=}\DecValTok{3}\NormalTok{)}
\end{Highlighting}
\end{Shaded}

\includegraphics{STA135-FinalProject_files/figure-latex/unnamed-chunk-17-1.pdf}

\begin{Shaded}
\begin{Highlighting}[]
\CommentTok{# since we reject the null, we use the simultaneous confidence intervals}
\CommentTok{# to check the significant components}
\end{Highlighting}
\end{Shaded}

\hypertarget{simultaneous-confidence-intervals}{%
\subsection{Simultaneous confidence
intervals}\label{simultaneous-confidence-intervals}}

\begin{Shaded}
\begin{Highlighting}[]
\NormalTok{wd<-}\KeywordTok{sqrt}\NormalTok{(cval}\OperatorTok{*}\KeywordTok{diag}\NormalTok{(Sp)}\OperatorTok{*}\KeywordTok{sum}\NormalTok{(}\DecValTok{1}\OperatorTok{/}\NormalTok{n))}
\NormalTok{Cis<-}\KeywordTok{cbind}\NormalTok{(d}\OperatorTok{-}\NormalTok{wd,d}\OperatorTok{+}\NormalTok{wd)}

\CommentTok{# 95% simultaneous confidence interval}
\NormalTok{Cis}
\end{Highlighting}
\end{Shaded}

\begin{verbatim}
##                     [,1]       [,2]
## AHF_activity  0.08499858 0.26115476
## AHF_antigen  -0.15649036 0.01275925
\end{verbatim}

\begin{Shaded}
\begin{Highlighting}[]
\CommentTok{#plot(Cis[1][1]:0, 1:10, type="l", lty=2)}
\end{Highlighting}
\end{Shaded}

\hypertarget{bonferroni-simultaneous-confidence-intervals}{%
\subsection{Bonferroni simultaneous confidence
intervals}\label{bonferroni-simultaneous-confidence-intervals}}

\begin{Shaded}
\begin{Highlighting}[]
\NormalTok{wd.b<-}\StringTok{ }\KeywordTok{qt}\NormalTok{(}\DecValTok{1}\OperatorTok{-}\NormalTok{alpha}\OperatorTok{/}\NormalTok{(}\DecValTok{2}\OperatorTok{*}\NormalTok{p),n[}\DecValTok{1}\NormalTok{]}\OperatorTok{+}\NormalTok{n[}\DecValTok{2}\NormalTok{]}\OperatorTok{-}\DecValTok{2}\NormalTok{) }\OperatorTok{*}\KeywordTok{sqrt}\NormalTok{(}\KeywordTok{diag}\NormalTok{(Sp)}\OperatorTok{*}\KeywordTok{sum}\NormalTok{(}\DecValTok{1}\OperatorTok{/}\NormalTok{n))}
\NormalTok{Cis.b<-}\KeywordTok{cbind}\NormalTok{(d}\OperatorTok{-}\NormalTok{wd.b,d}\OperatorTok{+}\NormalTok{wd.b)}
\CommentTok{# 95% Bonferroni simultaneous confidence interval}
\NormalTok{Cis.b}
\end{Highlighting}
\end{Shaded}

\begin{verbatim}
##                     [,1]        [,2]
## AHF_activity  0.09298751 0.253165822
## AHF_antigen  -0.14881465 0.005083538
\end{verbatim}

\begin{Shaded}
\begin{Highlighting}[]
\CommentTok{# both component-wise simultaneous confidence intervals do not contain 0, so they have significant differences.}
\end{Highlighting}
\end{Shaded}

\hypertarget{conclusion}{%
\subsection{CONCLUSION}\label{conclusion}}

\hypertarget{the-company-can-use-the-following-function-to-predict-the-amount-of-cpu-time-that-they-will-need-for-a-new-set-of-parameters}{%
\subsubsection{The company can use the following function to predict the
amount of CPU time that they will need for a new set of
parameters:}\label{the-company-can-use-the-following-function-to-predict-the-amount-of-cpu-time-that-they-will-need-for-a-new-set-of-parameters}}

\begin{itemize}
\tightlist
\item
  y = 1.07898250130454\(z_1\) + 0.419888473166963\(z_2\) +
  8.42368896741667

  \begin{itemize}
  \tightlist
  \item
    \(z_1\) is the value of the companies number of orders (in
    thousands) that they need to know CPU time requirements for
  \item
    \(z_2\) is the companies add-delete items (in thousands) that they
    need to know CPU time requirements for
  \item
    y is the predicted CPU time needed.
  \end{itemize}
\end{itemize}

\hypertarget{then-as-generated-in-the-last-portion-of-the-data-analysis-we-can-see-that-if-we-want-to-be-xx-where-an-alpha-value-of-0.05-corresponds-to-95-alpha-of-0.001-corresponds-to-99.9-etc.-sure-that-the-true-value-of-cpu-time-will-fall-within-the-lower-and-upper-bound-we-will-add-and-subtract-the-values-generated-to-the-value-of-y-predicted-above-for-some-value-of-z_01-being-the-orders-in-thousands-and-z_02-being-the-add-delete-items-in-thousands.}{%
\subsubsection{\texorpdfstring{Then, as generated in the last portion of
the data analysis, we can see that if we want to be XX\% (where an alpha
value of 0.05 corresponds to 95\%, alpha of 0.001 corresponds to 99.9\%
etc.) sure that the true value of CPU time will fall within the lower
and upper bound we will add and subtract the values generated to the
value of y predicted above for some value of \(z_{01}\) being the orders
in thousands and \(z_{02}\) being the add-delete items in
thousands.}{Then, as generated in the last portion of the data analysis, we can see that if we want to be XX\% (where an alpha value of 0.05 corresponds to 95\%, alpha of 0.001 corresponds to 99.9\% etc.) sure that the true value of CPU time will fall within the lower and upper bound we will add and subtract the values generated to the value of y predicted above for some value of z\_\{01\} being the orders in thousands and z\_\{02\} being the add-delete items in thousands.}}\label{then-as-generated-in-the-last-portion-of-the-data-analysis-we-can-see-that-if-we-want-to-be-xx-where-an-alpha-value-of-0.05-corresponds-to-95-alpha-of-0.001-corresponds-to-99.9-etc.-sure-that-the-true-value-of-cpu-time-will-fall-within-the-lower-and-upper-bound-we-will-add-and-subtract-the-values-generated-to-the-value-of-y-predicted-above-for-some-value-of-z_01-being-the-orders-in-thousands-and-z_02-being-the-add-delete-items-in-thousands.}}

\begin{itemize}
\tightlist
\item
  Alpha value: 0.05 Interval: 3.91241721099891
\item
  Alpha value: 0.01 Interval: 6.48784302704889
\item
  Alpha value: 0.005 Interval: 7.88779247898333
\item
  Alpha value: 0.001 Interval: 12.1331741930768
\end{itemize}

\hypertarget{in-addtion-these-values-can-automatically-be-generated-with-the-following-script-by-just-altering-the-line-z_0---as.matrixc11307.5-by-replacing-130-and-7.5-in-the-following-manner.-the-general-line-would-be-z_0---as.matrixc1z_01z_02-where-the-value-of-z_01-is-the-orders-in-thousands-and-z_02-is-the-add-delete-items-in-thousands.}{%
\paragraph{\texorpdfstring{In addtion, these values can automatically be
generated with the following script by just altering the line z\_0
\textless{}- as.matrix(c(1,130,7.5)) by replacing "130 and 7.5 in the
following manner. The general line would be z\_0 \textless{}-
as.matrix(c(1,\(z_{01}\),\(z_{02}\))) where the value of \(z_{01}\) is
the orders in thousands and \(z_{02}\) is the add-delete items in
thousands.}{In addtion, these values can automatically be generated with the following script by just altering the line z\_0 \textless{}- as.matrix(c(1,130,7.5)) by replacing "130 and 7.5 in the following manner. The general line would be z\_0 \textless{}- as.matrix(c(1,z\_\{01\},z\_\{02\})) where the value of z\_\{01\} is the orders in thousands and z\_\{02\} is the add-delete items in thousands.}}\label{in-addtion-these-values-can-automatically-be-generated-with-the-following-script-by-just-altering-the-line-z_0---as.matrixc11307.5-by-replacing-130-and-7.5-in-the-following-manner.-the-general-line-would-be-z_0---as.matrixc1z_01z_02-where-the-value-of-z_01-is-the-orders-in-thousands-and-z_02-is-the-add-delete-items-in-thousands.}}

\begin{verbatim}
for (val in c(0.05,0.01,0.005,0.001))
{
  z_0 <- as.matrix(c(1,130,7.5))
  z_0_beta_hat <- t(z_0)%*%beta_hat
  statistic<-qt(1-(val/2),n-r-1)
  ans<-t(z_0)%*%(solve(t(Z)%*%Z))%*%z_0
  interval <- sqrt(sigma_sq)*statistic*sqrt(1+ans)
  int_low <- z_0_beta_hat - interval
  int_up <- z_0_beta_hat + interval
  print (sprintf("Alpha value: %s  Interval: %s  Lower bound: %s  Upper bound: %s",  val, interval, int_low, int_up ))
}
\end{verbatim}

\hypertarget{as-a-precaution-we-do-suggest-gather-more-data-on-this-situation-if-possible.-this-will-increase-the-strength-of-the-analysis-and-account-for-other-variables-in-the-situation.-this-would-be-achieved-by-increasing-the-value-of-n-or-observed-situaitons-company-cpu-time-requirements-and-r-the-number-of-observed-varaibles-upon-which-to-make-the-prediction-of-cpu-time-orders-add-delete-items-etc..-please-contact-if-this-is-the-case-and-a-more-robust-script-will-be-made-to-factor-in-variations-in-new-potential-data-gathering.}{%
\subsubsection{As a precaution we do suggest gather more data on this
situation if possible. This will increase the strength of the analysis
and account for other variables in the situation. This would be achieved
by increasing the value of n, or observed situaitons (company CPU time
requirements), and r, the number of observed varaibles upon which to
make the prediction of CPU time (orders, add-delete items, etc.). Please
contact if this is the case and a more robust script will be made to
factor in variations in new potential data
gathering.}\label{as-a-precaution-we-do-suggest-gather-more-data-on-this-situation-if-possible.-this-will-increase-the-strength-of-the-analysis-and-account-for-other-variables-in-the-situation.-this-would-be-achieved-by-increasing-the-value-of-n-or-observed-situaitons-company-cpu-time-requirements-and-r-the-number-of-observed-varaibles-upon-which-to-make-the-prediction-of-cpu-time-orders-add-delete-items-etc..-please-contact-if-this-is-the-case-and-a-more-robust-script-will-be-made-to-factor-in-variations-in-new-potential-data-gathering.}}

\hypertarget{dataset-3-pca}{%
\section{DATASET 3: PCA}\label{dataset-3-pca}}

\hypertarget{introduction-1}{%
\subsection{INTRODUCTION:}\label{introduction-1}}

\hypertarget{data-has-been-gatherd-on-populations-which-is-considered-census-tract-data.-the-variables-that-have-been-gathered-are-total-population-thousands-professional-degree-employed-age-over-16-government-employment-median-home-value-100000s.}{%
\subsubsection{Data has been gatherd on populations which is considered
``Census-tract data''. The variables that have been gathered are ``Total
Population (thousands)'', ``Professional Degree (\%)'', ``Employed age
over 16 (\%)'', ``Government Employment (\%)'', ``Median home value
(\$100,000s)''.}\label{data-has-been-gatherd-on-populations-which-is-considered-census-tract-data.-the-variables-that-have-been-gathered-are-total-population-thousands-professional-degree-employed-age-over-16-government-employment-median-home-value-100000s.}}

\begin{enumerate}
\def\labelenumi{\arabic{enumi})}
\tightlist
\item
  Census data can be very important for a variety of reasons, one of the
  most important/common ones is predicting voting outcomes.
\item
  Politicians may try to predict their popularity to certain populations
  by find the more common types of districts and try to gain popularity
  with one of those districts which would then hopefully have a similar
  effect on those other common areas. \#\#\# PCA (principle component
  analysis) is a popular way to explore datasets with multiple
  dimensions due to the fact that it is a diminsional reduction
  technique which allows exploration of the data in 2 dimensions
  (depending on how much of the variance can be summarized in those two
  dimensions). This is a great high level method to explore which groups
  have the most variation and potentially cluster together; therefore,
  this data can help those interested, such as politicians, in
  understanding their demographic.
\end{enumerate}

\hypertarget{summarize-data-1}{%
\subsection{SUMMARIZE DATA:}\label{summarize-data-1}}

\hypertarget{lets-summarize-the-dataframe}{%
\subsubsection{Lets summarize the
dataframe}\label{lets-summarize-the-dataframe}}

\begin{verbatim}
##  Total_Population(thousands) Professional_Degree(%) Employed_age_over_16(%)
##  Min.   :1.360               Min.   : 0.720         Min.   :49.50          
##  1st Qu.:3.120               1st Qu.: 1.670         1st Qu.:66.42          
##  Median :4.720               Median : 3.380         Median :71.30          
##  Mean   :4.469               Mean   : 3.962         Mean   :71.42          
##  3rd Qu.:5.760               3rd Qu.: 4.830         3rd Qu.:77.33          
##  Max.   :9.210               Max.   :16.700         Max.   :86.54          
##  Government_Employment(%) Median_home_value($100,000s)
##  Min.   :16.30            Min.   :0.930               
##  1st Qu.:20.60            1st Qu.:1.300               
##  Median :24.40            Median :1.490               
##  Mean   :26.91            Mean   :1.636               
##  3rd Qu.:31.00            3rd Qu.:1.780               
##  Max.   :68.50            Max.   :3.640
\end{verbatim}

\begin{verbatim}
## [1] 61  5
\end{verbatim}

\hypertarget{lets-also-look-at-some-box-plots-to-help-visualize-the-distribution-of-our-varibles}{%
\subsubsection{Lets also look at some box plots to help visualize the
distribution of our
varibles}\label{lets-also-look-at-some-box-plots-to-help-visualize-the-distribution-of-our-varibles}}

\includegraphics{STA135-FinalProject_files/figure-latex/unnamed-chunk-21-1.pdf}
\includegraphics{STA135-FinalProject_files/figure-latex/unnamed-chunk-21-2.pdf}
\includegraphics{STA135-FinalProject_files/figure-latex/unnamed-chunk-21-3.pdf}
\includegraphics{STA135-FinalProject_files/figure-latex/unnamed-chunk-21-4.pdf}
\includegraphics{STA135-FinalProject_files/figure-latex/unnamed-chunk-21-5.pdf}

\hypertarget{data-analysis-1}{%
\subsection{DATA ANALYSIS:}\label{data-analysis-1}}

\hypertarget{first-lets-take-a-look-at-the-eigen-values-and-eigen-vectors-of-the-covariance-matrix.}{%
\subsubsection{First lets take a look at the eigen values and eigen
vectors of the covariance
matrix.}\label{first-lets-take-a-look-at-the-eigen-values-and-eigen-vectors-of-the-covariance-matrix.}}

\begin{verbatim}
## [1] 107.0152535  39.6721358   8.3708660   2.8678740   0.1546931
\end{verbatim}

\begin{verbatim}
##              [,1]        [,2]        [,3]        [,4]         [,5]
## [1,]  0.038887287 -0.07114494 -0.18789258  0.97713524 -0.057699864
## [2,] -0.105321969 -0.12975236  0.96099580  0.17135181 -0.138554092
## [3,]  0.492363944 -0.86438807 -0.04579737 -0.09104368  0.004966048
## [4,] -0.863069865 -0.48033178 -0.15318538 -0.02968577  0.006691800
## [5,] -0.009122262 -0.01474342  0.12498114  0.08170118  0.988637470
\end{verbatim}

\hypertarget{also-lets-take-a-look-at-the-eigen-values-and-eigen-vectors-of-the-correlation-matrix-which-is-also-sometimes-a-good-way-to-summarize-our-data-can-be-preferrable-if-the-primary-eigen-values-are-a-larger-proportion-than-that-of-the-covariance-matrix.}{%
\subsubsection{Also, lets take a look at the eigen values and eigen
vectors of the correlation matrix which is also sometimes a good way to
summarize our data (can be preferrable if the primary eigen values are a
larger proportion than that of the covariance
matrix).}\label{also-lets-take-a-look-at-the-eigen-values-and-eigen-vectors-of-the-correlation-matrix-which-is-also-sometimes-a-good-way-to-summarize-our-data-can-be-preferrable-if-the-primary-eigen-values-are-a-larger-proportion-than-that-of-the-covariance-matrix.}}

\begin{verbatim}
## [1] 1.9919183 1.3675266 0.8641573 0.5350610 0.2413367
\end{verbatim}

\begin{verbatim}
##            [,1]       [,2]        [,3]       [,4]       [,5]
## [1,]  0.2625829  0.4629936  0.78390268  0.2169291  0.2347882
## [2,] -0.5933541  0.3256442 -0.16407255 -0.1446471  0.7028828
## [3,]  0.3256978  0.6051419 -0.22487455 -0.6628689 -0.1943206
## [4,] -0.4792022 -0.2524850  0.55070086 -0.5716730 -0.2766497
## [5,] -0.4932213  0.4996473 -0.06882436  0.4072024 -0.5801162
\end{verbatim}

\hypertarget{lets-see-if-the-covariance-matrix-or-the-correlation-matrix-summarizes-the-data-in-the-first-two-components-better.}{%
\subsubsection{Lets see if the covariance matrix or the correlation
matrix summarizes the data in the first two components
better.}\label{lets-see-if-the-covariance-matrix-or-the-correlation-matrix-summarizes-the-data-in-the-first-two-components-better.}}

\begin{verbatim}
## [1] "Variance summarized in first two components of the covariance matrix:"
\end{verbatim}

\begin{verbatim}
## [1] 0.9279265
\end{verbatim}

\begin{verbatim}
## [1] "Variance summarized in first two components of the corrlation matrix:"
\end{verbatim}

\begin{verbatim}
## [1] 0.671889
\end{verbatim}

\hypertarget{compare-to-the-standardized-matrix}{%
\subsection{COMPARE TO THE STANDARDIZED
MATRIX}\label{compare-to-the-standardized-matrix}}

\begin{verbatim}
##              [,1]        [,2]         [,3]        [,4]        [,5]
##  [1,] -0.97609885  0.56194340 -0.321768114  0.35870114 -0.27561091
##  [2,] -1.20397978  0.13109026  0.209185541  1.73618299 -0.34647398
##  [3,] -0.73194072  2.02813021 -0.868811274  0.53883338  0.84048257
##  [4,]  0.36405801  1.11819410 -0.017408317 -0.25586769  0.37987256
##  [5,]  0.58108746  1.70016737  0.471980785  0.43287324  1.05307180
##  [6,]  0.30980065  0.28221039 -2.387928677  2.25538769 -0.06302167
##  [7,] -0.72108925  0.27577974 -0.592608110  1.13221018 -0.20474783
##  [8,] -1.10631652 -0.50232817 -0.565792269  1.04744206 -0.41733706
##  [9,]  0.49427568  0.10858300  1.556681561 -0.76447637  0.76961949
## [10,]  1.55771999 -0.39622254  0.158235443 -0.25586769 -0.38190552
## [11,]  0.25554328  0.22433459 -0.951940382  0.08320477 -0.38190552
## [12,]  0.19043445  0.09572172  1.504390671 -0.70090029 -0.31104244
## [13,]  0.29894918  0.06678382  1.720258192 -0.66911224 -0.38190552
## [14,] -0.59629731 -0.95247324 -0.199756037 -1.11414484 -0.82479976
## [15,] -0.45522817  0.78380061 -0.148805938  0.22095295  0.64560911
## [16,]  1.60655162  0.65518773 -0.119308513  1.14280619 -0.34647398
## [17,] -1.23110846  2.13102051 -0.210482373  1.56664676  0.66332488
## [18,]  1.46005674  0.24041120  1.075337212  0.64479352 -0.15160052
## [19,]  1.00972063 -0.34799271 -0.694508306  1.18519025  0.16728333
## [20,] -1.03035622 -0.67917087 -0.559088308  0.68717758 -0.80708399
## [21,]  1.03684931 -0.77241521 -1.128924934 -0.92341659 -1.24997823
## [22,]  0.47257274 -0.17758065  0.154213067 -0.72209232  0.04327295
## [23,]  0.21756313  0.39796198  0.497455834 -1.10354883  3.55099530
## [24,] -1.31792024  0.27899507 -0.488026329 -1.00818470 -0.25789514
## [25,]  0.54853304 -0.84315229  0.805838008 -0.56315210 -0.55906322
## [26,]  1.79102666  0.43654584 -1.722895816  0.45406527  2.78921721
## [27,]  0.98801768 -0.43802173 -0.951940382  0.05141672  0.25586218
## [28,]  1.03684931 -0.08112099  0.964051471  0.76134968 -0.59449476
## [29,]  0.70045366  0.03141528  1.655900173  0.47525729 -0.20474783
## [30,]  0.84694854 -0.27725563  0.619467912 -0.20288762 -0.98424168
## [31,]  0.33692933 -0.67917087  0.433097815 -0.29825175 -1.17911515
## [32,] -0.05914942 -0.73704667 -0.803112463 -0.34063580 -1.00195745
## [33,] -0.75906940 -0.63094104 -0.454506528 -0.07573545 -0.78936822
## [34,] -1.43728644 -0.91067405 -2.938994214 -0.53136406 -0.02759013
## [35,] -0.62885173 -0.97176517  0.446505736 -0.04394740 -0.91337861
## [36,] -0.55289142 -1.04250225 -0.728028108 -0.52076804 -0.77165245
## [37,] -1.48069233 -0.96211921 -1.498983542 -0.52076804 -0.82479976
## [38,] -1.44271217 -0.77884585 -0.183666532 -0.26646370 -1.12596784
## [39,]  0.60821614 -0.74026199  0.876899987 -1.03997274 -0.59449476
## [40,] -0.40639654 -0.73061603  1.472211661 -1.12474086 -0.20474783
## [41,] -0.58544584 -0.87530551 -0.569814645  0.08320477 -1.07282053
## [42,] -1.20397978 -0.37371529 -0.081766335 -0.37242385 -0.87794707
## [43,] -0.62885173 -0.85601358 -0.016067525 -0.81745644 -0.75393668
## [44,]  0.43459258 -0.88816680  0.222593462  0.35870114 -0.50591591
## [45,] -0.65598041 -0.94604260  0.394214846 -1.10354883 -0.71850514
## [46,]  1.24302729 -0.78206118  0.931872461 -0.45719196 -0.54134745
## [47,] -0.84045545  0.14716687 -1.732281360  4.40637858  1.08850334
## [48,] -1.53494969  4.09558222 -0.913057412  2.38253986  2.64749105
## [49,] -1.68687031  3.31104366 -0.670374049 -0.46778797  2.06287066
## [50,] -0.48235685 -0.18722662 -0.784341374 -0.08633146 -0.57677899
## [51,] -0.59087158 -0.57628057 -0.713279395 -0.45719196 -0.34647398
## [52,]  1.50888837 -0.90102809  0.951984342 -0.35123182 -0.24017937
## [53,]  0.52683010 -0.33191610  0.290973857 -0.48898000  0.02555718
## [54,]  0.73843381  0.16324348  0.792430087 -0.07573545  0.92906142
## [55,] -0.39554507 -0.54734267  1.110197805 -0.71149630 -0.09845321
## [56,]  2.57233268 -0.51518945  0.423712271 -0.54196007  0.14956756
## [57,] -1.26366288  0.75164739  2.027299574 -1.00818470  2.06287066
## [58,]  1.16706698  0.26613378  0.994889688 -0.73268833  1.23022950
## [59,] -0.12425826  0.59731194 -0.004000396  0.01962868  0.09642025
## [60,]  0.13617709  0.24041120  0.883603947 -0.66911224 -0.15160052
## [61,]  1.09110667  0.31114829  0.376784549 -0.63732420  0.61017757
## attr(,"scaled:center")
## [1]  4.469016  3.962295 71.419836 26.914754  1.635574
## attr(,"scaled:scale")
## [1] 1.8430678 3.1101085 7.4582781 9.4375109 0.5644688
\end{verbatim}

\begin{verbatim}
## [1] 1.9919183 1.3675266 0.8641573 0.5350610 0.2413367
\end{verbatim}

\begin{verbatim}
##            [,1]       [,2]        [,3]       [,4]       [,5]
## [1,]  0.2625829  0.4629936 -0.78390268  0.2169291  0.2347882
## [2,] -0.5933541  0.3256442  0.16407255 -0.1446471  0.7028828
## [3,]  0.3256978  0.6051419  0.22487455 -0.6628689 -0.1943206
## [4,] -0.4792022 -0.2524850 -0.55070086 -0.5716730 -0.2766497
## [5,] -0.4932213  0.4996473  0.06882436  0.4072024 -0.5801162
\end{verbatim}

\hypertarget{also-lets-take-a-look-at-the-eigen-values-and-eigen-vectors-of-the-correlation-matrix-which-is-also-sometimes-a-good-way-to-summarize-our-data-can-be-preferrable-if-the-primary-eigen-values-are-a-larger-proportion-than-that-of-the-covariance-matrix.-1}{%
\subsubsection{Also, lets take a look at the eigen values and eigen
vectors of the correlation matrix which is also sometimes a good way to
summarize our data (can be preferrable if the primary eigen values are a
larger proportion than that of the covariance
matrix).}\label{also-lets-take-a-look-at-the-eigen-values-and-eigen-vectors-of-the-correlation-matrix-which-is-also-sometimes-a-good-way-to-summarize-our-data-can-be-preferrable-if-the-primary-eigen-values-are-a-larger-proportion-than-that-of-the-covariance-matrix.-1}}

\begin{verbatim}
## [1] 1.9919183 1.3675266 0.8641573 0.5350610 0.2413367
\end{verbatim}

\begin{verbatim}
##            [,1]       [,2]        [,3]       [,4]       [,5]
## [1,]  0.2625829  0.4629936  0.78390268  0.2169291  0.2347882
## [2,] -0.5933541  0.3256442 -0.16407255 -0.1446471  0.7028828
## [3,]  0.3256978  0.6051419 -0.22487455 -0.6628689 -0.1943206
## [4,] -0.4792022 -0.2524850  0.55070086 -0.5716730 -0.2766497
## [5,] -0.4932213  0.4996473 -0.06882436  0.4072024 -0.5801162
\end{verbatim}

\hypertarget{lets-see-if-the-covariance-matrix-or-the-correlation-matrix-summarizes-the-data-in-the-first-two-components-better.-1}{%
\subsubsection{Lets see if the covariance matrix or the correlation
matrix summarizes the data in the first two components
better.}\label{lets-see-if-the-covariance-matrix-or-the-correlation-matrix-summarizes-the-data-in-the-first-two-components-better.-1}}

\begin{verbatim}
## [1] "Variance summarized in first two components of the covariance matrix:"
\end{verbatim}

\begin{verbatim}
## [1] 0.9279265
\end{verbatim}

\begin{verbatim}
## [1] "Variance summarized in first two components of the corrlation matrix:"
\end{verbatim}

\begin{verbatim}
## [1] 0.671889
\end{verbatim}

\hypertarget{so-we-see-that-using-the-covariance-matrix-is-preferred-here.}{%
\subsubsection{So we see that using the covariance matrix is preferred
here.}\label{so-we-see-that-using-the-covariance-matrix-is-preferred-here.}}

\hypertarget{so-since-we-can-summarize-92-of-the-variance-in-our-data-using-the-first-two-principal-components-it-would-fair-to-graph-our-analysis-using-these-two-pcs.-lets-also-looks-the-the-eigen-value-size-graphed-over-the-five-components.-finally-we-will-overlay-the-pc-scores-for-the-sample-data-in-the-space-of-the-first-two-principal-components-so-we-can-visualize-which-of-the-sample-data-is-most-contributing-to-the-the-principal-components-which-we-see-government-employment-percentage-and-employed-age-over-16-are-the-main-two-contributors-to-the}{%
\subsubsection{So since we can summarize 92\% of the variance in our
data using the first two principal components it would fair to graph our
analysis using these two PCs. Lets also looks the the eigen value size
graphed over the five components. Finally, we will overlay the PC scores
for the sample data in the space of the first two principal components
so we can visualize which of the sample data is most contributing to the
the principal components which we see Government employment percentage
and employed age over 16 (\%) are the main two contributors to
the}\label{so-since-we-can-summarize-92-of-the-variance-in-our-data-using-the-first-two-principal-components-it-would-fair-to-graph-our-analysis-using-these-two-pcs.-lets-also-looks-the-the-eigen-value-size-graphed-over-the-five-components.-finally-we-will-overlay-the-pc-scores-for-the-sample-data-in-the-space-of-the-first-two-principal-components-so-we-can-visualize-which-of-the-sample-data-is-most-contributing-to-the-the-principal-components-which-we-see-government-employment-percentage-and-employed-age-over-16-are-the-main-two-contributors-to-the}}

\hypertarget{todo-how-do-we-do-this-with-stuff-we-have-learned-in-the-course.}{%
\paragraph{TODO, how do we do this with stuff we have learned in the
course.}\label{todo-how-do-we-do-this-with-stuff-we-have-learned-in-the-course.}}

\hypertarget{todo-do-we-need-to-do-this-for-the-standardized-matrix-instead}{%
\paragraph{TODO, Do we need to do this for the standardized matrix
instead?}\label{todo-do-we-need-to-do-this-for-the-standardized-matrix-instead}}

\begin{verbatim}
## Importance of components:
##                            Comp.1    Comp.2     Comp.3     Comp.4       Comp.5
## Standard deviation     10.2596737 6.2467410 2.86943177 1.67954149 0.3900732431
## Proportion of Variance  0.6769654 0.2509611 0.05295308 0.01814182 0.0009785696
## Cumulative Proportion   0.6769654 0.9279265 0.98087961 0.99902143 1.0000000000
## 
## Loadings:
##                              Comp.1 Comp.2 Comp.3 Comp.4 Comp.5
## Total_Population(thousands)                 0.188  0.977       
## Professional_Degree(%)       -0.105  0.130 -0.961  0.171  0.139
## Employed_age_over_16(%)       0.492  0.864                     
## Government_Employment(%)     -0.863  0.480  0.153              
## Median_home_value($100,000s)               -0.125        -0.989
\end{verbatim}

\begin{verbatim}
##      Comp.1      Comp.2      Comp.3      Comp.4      Comp.5 
## 105.2609051  39.0217730   8.2336387   2.8208596   0.1521571
\end{verbatim}

\includegraphics{STA135-FinalProject_files/figure-latex/unnamed-chunk-28-1.pdf}
\includegraphics{STA135-FinalProject_files/figure-latex/unnamed-chunk-28-2.pdf}
\includegraphics{STA135-FinalProject_files/figure-latex/unnamed-chunk-28-3.pdf}

\hypertarget{conclusion-1}{%
\subsection{CONCLUSION}\label{conclusion-1}}

\hypertarget{as-a-precaution-we-do-suggest-gather-more-data-on-this-situation-if-possible.-this-will-increase-the-strength-of-the-analysis-and-account-for-other-variables-in-the-situation.-this-would-be-achieved-by-increasing-the-value-of-n-or-observed-situaitons-census-data-in-new-areas-and-r-the-number-of-observed-varaibles-upon-which-to-build-pcs-such-as-total_populationthousands-professional_degree-and-employed_age_over_16-etc.-please-contact-if-this-is-the-case-and-a-more-robust-script-will-be-made-to-factor-in-variations-in-new-potential-data-gathering.}{%
\subsubsection{As a precaution we do suggest gather more data on this
situation if possible. This will increase the strength of the analysis
and account for other variables in the situation. This would be achieved
by increasing the value of n, or observed situaitons (census data in new
areas), and r, the number of observed varaibles upon which to build PCs
(such as ``Total\_Population(thousands)'', ``Professional\_Degree(\%)'',
and ``Employed\_age\_over\_16(\%)'' etc). Please contact if this is the
case and a more robust script will be made to factor in variations in
new potential data
gathering.}\label{as-a-precaution-we-do-suggest-gather-more-data-on-this-situation-if-possible.-this-will-increase-the-strength-of-the-analysis-and-account-for-other-variables-in-the-situation.-this-would-be-achieved-by-increasing-the-value-of-n-or-observed-situaitons-census-data-in-new-areas-and-r-the-number-of-observed-varaibles-upon-which-to-build-pcs-such-as-total_populationthousands-professional_degree-and-employed_age_over_16-etc.-please-contact-if-this-is-the-case-and-a-more-robust-script-will-be-made-to-factor-in-variations-in-new-potential-data-gathering.}}

\end{document}
